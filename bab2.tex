%!TEX root = ./template-skripsi.tex
%-------------------------------------------------------------------------------
%                            BAB II
%               TINJAUAN PUSTAKA DAN DASAR TEORI
%-------------------------------------------------------------------------------

\chapter{TINJAUAN PUSTAKA DAN DASAR TEORI}                

\section{Telegram Bot}
Telegram adalah aplikasi pesan chatting yang memungkinkan pengguna untuk
mengirimkan pesan chatting rahasia yang dienkripsi end-to-end sebagai keamanan
tambahan. Selain itu telegram dilengkapi dengan fitur bot yang bersifat open source
yang dimana fitur ini sangat cocok digunakan dalam penelitian ini. pada penelitian
ini penulis menggunakan aplikasi telegram dikarenakan fitur bot telegram bisa digu-
nakan sebagai \emph{Command and Control (CNC)} pada \emph{server} , baik \emph{memonitoring server}
, monitoring serangan. Namun pada penelitian ini bot difungsikan sebagai Command
and Control pada server yaitu jika terjadi serangan apakah alamat ipaddres yang ter-
deteksi akan di blok atau tidak. Sehingga sysadmin tidak perlu melakukan remote
server untuk itu[2].

\section {\emph{Intrusion Detection System (IDS)}}
Menurut Chris Brenton (2003:289), Intrusion Detection System (IDS) ada-
lah sistem pendeteksian penyusupan yang dapat melakukan scanning log-log access
dan menganalisis karakteristik-karakteristik dari file-file untuk mengetahui apakah fi-
le tersebut telah diserang. Intrusion Detection System mampu mendeteksi aktivitas
yang mencurigakan dalam sebuah sistem atau jaringan.[1]

\indent
\subsection {\emph {Network Intrusion Detection System (\emph{NIDS})}}
\emph{Network Intrusion Detection System} adalah sistem pendeteksian penyusupan
\emph{network}. \emph{network Intrusion Detection System} memiliki fungsi untuk menganalisis
paket di sebuah \emph{network} dan mencoba untuk menentukan apakah seorang cracker
sedang mencoba untuk masuk ke dalam sebuah sistem atau menyebabkan sebuah
serangan Denial of Service (DOS)[2].

\subsection{\emph{Host Intrusion Detection System (\emph{HIDS})}}
\emph{Host Intrusion Detection System} adalah sistem pendeteksian penyusupan host.
Sama seperti \emph{NIDS}, sebuah \emph{HIDS} menganalisis lalu lintas \emph{network} yang dikirimkan
menuju dan dari sebuah mesin tunggal. Sebagian besar dari \emph{NIDS} komersial saat ini
biasanya memiliki suatu unsur \emph{HIDS}, dan sistem-sistem ini disebut hybrid IDS[2].

\subsection{\emph{System Integrity Verifier (SIV)}}
\emph{System Integrity Verifier} adalah alat untuk verifikasi integritas sistem. Melacak
file-file sistem yang kritikal dan memberitahukan kepada administrator pada saat file-
file tersebut diubah (biasanya oleh seorang cracker yang mencoba untuk mengganti
file yang valid dengan sebuah Trojan Horse). Contoh dari SIV adalah Tripwire[3].

\subsection{\emph{Log File Monitor (LFM)}}
Log File Monitor adalah alat yang digunakan untuk membaca log-log yang
dihasilkan oleh servis- servis \emph{network} yang mencari pola-pola serangan. Contoh dari
LFM adalah Swatch[3].

\section{\emph{Distributed Denial of Service (DDoS)}}
Rui Zhong et al.[4] \emph{Distributed Denial of Service (DDoS)} adalah jenis se-
rangan yang dilakukan secara masif dengan tujuan mengganggu hak akses pengguna
jaringan. DDoS merupakan serangan flooding trafik yang dilakukan dengan sengaja
untuk mengganggu QoS dari sistem jaringan yang bertujuan untuk membuat sumber
daya server habis. Serangan DDoS pada dasarnya sama dengan serangan DoS namun
serangan dilakukan dengan banyak sumber secara serentak. Untuk melancarkan se-
rangan DDoS, penyerang biasanya mengumpulkan pasukan dengan cara mengambil
alih komputer-komputer yang kemudian dijadikan zombie yang merupakan komputer
yang siap diperintah dan dikendalikan oleh botnet[2].

\newpage 
\section{\emph{Brute Force Attack}}
\emph{Brute Force attack} adalah serangan pada protokol jaringan yang bertujuan un-
tuk mendapat hak akses penuh dengan cara melakukan kegiatan penebakan username
dan password login dari sebuah service dengan menggunakan kombinasi username
dan password yang berbeda[4] 


\section{\emph{Scanning}}
\emph{Scanning Attack} adalah serangan yang bertujuan pada jaringan yang bertujuan
untuk menemukan port ataupun service yang terdapat pada sebuah host yang sedang
tersambung kedalam jaringan[5].Berikut akan dijelaskan macam-macam serangan
scanning:

\subsection{\emph{Host Discovery}}
Serangan \emph{Host Discovery} adalah jenis serangan scaning yang digunakan untuk
mengetahui junmlah host yang sedang aktip pada suatu jaringan[5]. Jenis serangan
ini mengirimkan packet icmp kesetiap alamat subnet pada sebauh jaringan.


\subsection{\emph{Port Detections}}
Jenis serangan scanning ini bertujuan untuk mengetahui port-port yang sedang
terbuka pada sebuah host yang telah ditentukan. Untuk menemukan port yang terbu-
ka serangan ini melakukan pengiriman paket tcp terhadap host yang sedanga aktip
dengan menargetkan port 1-65536.

\subsection{\emph{Service Scanning}}
Jenis serangan scanning ini adalah serangan yang akan mengetahui macam
macam service yang berjalan pada sebuah host yang menjalankan port tertentu.Serangan
ini melakukan scanning pada masing-masing port yang sedang berjalan dan akan me-
lakukan penyamaan signature pada software yang menjalankan port tersebut[5].

\newpage
\subsection{\emph{Host Detection}}
Serangan scanning ini bertujuan untuk mengetahui jenis sistem operasi yang
dijalankan. Serangan ini menargetkan port 443 (netbios) yang mempunyai signature
(ciri-ciri) tiap masing-masing host[6].

\subsection{\emph{Scapy}}
Scapy adalah suatu library yang dibangun menggunakan bahasa pemrogram-
an python dan sangat powerfull untuk memanipulasi paket yang ada pada jaringan.
Scapy mampu membuat dan memecah paket-paket dari berbagai jenis protocol yang
ada, mentransimikannya, menangkapnya, menerima permintaan dan menjawabnya,
dan banyak lagi. Scapy dapat digunakan untuk menangani macam-macam kegiat-
an yang berhubungan dengan \emph{network}ing, seperti kegiatan "scanning","tracerouting",
"attack" atau "\emph{network} discovery". Scapy juga dapat mengajarkan kita tentang semua
proses-proses dari suatu protocol.


% Baris ini digunakan untuk membantu dalam melakukan sitasi
% Karena diapit dengan comment, maka baris ini akan diabaikan
% oleh compiler LaTeX.
\begin{comment}
\bibliography{daftar-pustaka}
\end{comment}
