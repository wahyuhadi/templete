%!TEX root = ./template-skripsi.tex
%-------------------------------------------------------------------------------
% 								BAB I
% 							LATAR BELAKANG
%-------------------------------------------------------------------------------

\chapter{LATAR BELAKANG}

\section{Latar Belakang Masalah}
Dalam perkembangan teknologi yang semakin pesat, kebutuhan akan keaman-
an jaringan tentunya meningkat seiring dengan berkembangnya ilmu pengetahuan
tentang masalah \emph{hacking} dan \emph{cracking} yang bersifat \emph{free} dan ada pula yang diko-
mersilkan. Kemudian dari sisi \emph{software} pendukung pun sudah banyak \emph{tools} yang
bersifat \emph{free} yang kemampuannya sudah bisa dikatakan mumpuni untuk digunakan
sebagai alat penyerangan oleh kalangan \emph{attacker}.

Pada sisi lain timbul masalah serius yaitu faktor keamanannya, namun disatu
sisi manusia sudah sangat tergantung dengan sistem informasi. Hal itu yang me-
nyebabkan statistik insiden keamanan jaringan terus meningkat tajam dari tahun ke
tahun. Ini disebabkan karena kepedulian masyarakat yang sangat kurang terhadap
sistem keamanan jaringan

Keamanan jaringan lokal ini bergantung sepenuhnya terhadap bagaimana se-
orang \emph{network administrator} merespon dengan cepat sebuah serangan yang terjadi.
Tapi \emph{network administrator} hanyalah seorang manusia yang terbatas akan waktu. Se-
orang \emph{network administrator} tidak dapat mengawasi seluruh jaringan secara terus-
menerus. Maka dari itu dibutuhkan sebuah sistem yang dapat membantu \emph{network
administrator} untuk digunakan sebagai mengetasi segala macam serangan.
Pada permasalahan tersebut, pada penelitian ini akan dibuat sebuah aplikasi
yang dapat membantu \emph{network administrator} dalam \emph{memonitoring server}, aplikasi ini
bertujuan untuk mempermudah \emph{network administrator} dalam mengamankan \emph{server}
dari berbagai macam jenis serangan \emph{(ddos, scaning , brute force)}.
 
Selain itu aplikasi ini terhubung dengan fitur bot yang dimiliki oleh aplika-
si \emph{chat} telegram, yang berfungsi sebagai \emph{command and control} pada server. Setiap
serangan yang terdeteksi akan dikirim melalui telegram, sehingga \emph{network adminis-
trator} dapat mengetahui serangan apa saja yang terjadi pada \emph{server} ditambah dengan
fitur bot dari telegram yang berfungsi sebagai \emph{command and control} yang digunakan
untuk memerintah \emph{server} untuk melakukan pencegahan / \emph{bloking}.

\section{Rumusan Masalah}
Dalam perkembangan teknologi sekarang yang sudah semakin pesat, kebu-
tuhan akan keamanan jaringan tentunya meningkat seiring dengan berkembangnya
ilmu pengetahuan tentang masalah \emph{hacking} dan \emph{cracking} yang bersifat free dan ada
pula yang dikomersilkan. Kemudian dari sisi software pendukung pun sudah banyak
tool-tool yang bersifat free yang kemampuannya sudah bisa dikatakan mumpuni un-
tuk digunakan sebagai alat penyerangan oleh kalangan attacker. Serangan-serangan
tersebut dapat melumpuhkan server. Sehingga dapat menimbulkan kerugian.


\section{Batasan Masalah}
Batasan masalah pada penelitian ini adalah:
\begin{enumerate}
\item Membutuhkan kapasitas memori yang cukup besar.
\item Berjalan sistem operasi linux .
\item Hanya bisa melakukan monitoring pada satu \emph{interface network} saja..
\item Hanya mendeteksi serangan (DDoS, Scanning, Brute Force)
\end{enumerate}

\section{Hipotesis}
Dengan menggunakan rule deteksi yang didapatkan dengan pengolahan data-
sheet akan ditemukan fitur serangan yang mempunyai ciri-ciri masing-masing dalam
jenis serangan, hal ini akan mempermudah dalam melakukan deteksi serangan ter-
sebut, dikarenakan semua jenis serangan akan dibedakan dari masing masing fitur
serangan yang telah ditentukan. Dalam mendeteksi serangan diharapkan sekurang-kurangnya 70\% 


\section{Tujuan Penelitian}
Tujuan dari penelitian tugas akhir ini adalah membuat sebuah aplikasi yang digunakan untuk membantu \emph{sysadmin} dalam memonitoring serangan-serangan yang terjadi pada \emph{server}, baik dalam proses pencegahan dan pendektesian terhadap serangan yang mempu membahayakan \emph{server}.



\section{Metode Penyelesain Masalah}

Metode penelitian yang digunakan:
\begin{enumerate}

\item  STUDI LITERATUR. \newline
Melakukan pencarian refrensi mengenai telegram bot dan pengolahan data tra-
fik jaringan berdasarkan serangan yang diperlukan.

\item PENGUMPULAN DATA. \newline
Pada tahap ini, dilakukan pengumpulan data training yang akan diolah meng-
gunakan algoritma Decision Tree. Data training dikumpulkan menggunakan
scapy. Data Training pada serangan DDoS, Brute Force dan Scanning masing-
masing berjumlah 5 juta data.

\item PERANCANGAN KEBUTUHAN SISTEM. \newline
Melakukan perancangan sistem deteksi untuk mendeteksi serangan DDoS, Bru-
te Force dan Scanning serta dapat di integrasikan terhadap library scapy


\item PENGUJIAN SISTEM. \newline
Pada tahap ini sistem yang telah dibangun akan diuji berdasarkan hasil analisa
dari algoritma decision tree yang menghasilkan fitur dan rules serangan..Hasil
dari pengujian tersebut, diantaranya adalah kemampuan sistem untuk mengha-
silkan tree berdasarkan jumlah data training yang ditentukan dan kemampuan
sistem untuk mendeteksi serangan berdasarkan fitur dan rules serangan yang di
inputkan kedalam sistem deteksi.

\item ANALISA HASIL PENGUJIAN. \newline
Pada tahap analisis hasil pengujian, dilakukan perbandingan trafik serangan
terhadap jumlah keseluruhan trafik. Hasil dari analisis tersebut, diantaranya
adalah akurasi untuk mendeteksi serangan.

\item PENYUSUNAN LAPORAN TUGAS AKHIR. \newline
Pada tahap ini semua data dan hasil dari penelitian akan dibuat menjadi sebuah
laporan dengan sistematika penulisan yang sesuai dengan ketentuan institusi.

\end{enumerate}
\newpage
\section{Sistematika Penulisan}
\noindent
\textbf{BAB I : PENDAHULUAN}

Pada bab ini dijelaskan latar belakang, rumusan masalah, batasan, tujuan,
manfaat, keaslian penelitian, dan sistematika penulisan.\\


\noindent
\textbf{BAB II : TINJAUAN PUSTAKA DAN LANDASAN TEORI}

Pada bab ini dijelaskan teori-teori dan penelitian terdahulu yang digunakan
sebagai acuan dan dasar dalam penelitian.\\

\noindent
\textbf{BAB III : METODOLOGI PENELITIAN}

Pada bab ini dijelaskan metode yang digunakan dalam penelitian meliputi
langkah kerja, pertanyaan penilitian, alat dan bahan, serta tahapan dan alur penelitian.\\

\noindent
\textbf{BAB IV : HASIL DAN PEMBAHASAN}

Pada bab ini dijelaskan hasil penelitian dan pembahasannya.\\

\noindent
\textbf{BAB V : KESIMPULAN DAN SARAN}

Pada bab ini ditulis kesimpulan akhir dari penelitian dan saran untuk pengem-
bangan penelitian selanjutnya.\\


% Baris ini digunakan untuk membantu dalam melakukan sitasi
% Karena diapit dengan comment, maka baris ini akan diabaikan
% oleh compiler LaTeX.
\begin{comment}
\bibliography{daftar-pustaka}
\end{comment}
