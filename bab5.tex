%!TEX root = ./template-skripsi.tex
%-------------------------------------------------------------------------------
%                            	BAB V
%               		KESIMPULAN DAN SARAN
%-------------------------------------------------------------------------------

\chapter{KESIMPULAN DAN SARAN}

\section{Kesimpulan}
	Berdasarkan hasil analisis dan pengujian fungsional aplikasi ini, didapat kesimpulan sebagai berikut:

	\begin{enumerate}
		\item Semakin banyak data yang diolah sebagai rule deteksi semakin bagus akurasi
		yang didapatkan.
		
		\item Hasil akurasi yang tidak menggunakan data normal lebih besar dari pada menggunakan data normal.
		
		\item Akurasi deteksi scanning , brute force dan DDoS diatas 75\%.
		
		\item  Waktu rata-rata komunikasi server dan telegram dibawah 2 detik.
		
	
		
	\end{enumerate}


\section{Saran}
	\begin{enumerate}
		\item Gunakan \emph{server} yang mempunyai kapasitas memory yang besar . 
		
		\item Diharapkan dapat melakukan pengolahan data secara realtime sehinnga dijadikan \emph{software} dengan sistem \emph{autonomous system} yang dapat melakukan update rule secara berkala. 
		
		\item Dengan dasar sofware ini yang dibuat dengan rule terpisah dengan program inti , dihrapkan pengambang dapat mendeteksi jenis serangan lainya.
 
	\end{enumerate}

	
% Baris ini digunakan untuk membantu dalam melakukan sitasi
% Karena diapit dengan comment, maka baris ini akan diabaikan
% oleh compiler LaTeX.
\begin{comment}
\bibliography{daftar-pustaka}
\end{comment}
