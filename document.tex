\documentclass[11pt,a4paper,sans,french]{moderncv}        % autres options possibles : taille de fonte ('10pt', '11pt' et '12pt'), format de papier ('a4paper', 'letterpaper', 'a5paper', 'legalpaper', 'executivepaper' and 'landscape') et famille de fonte ('sans' and 'roman')
\moderncvstyle{casual}                             % autres styles : 'casual' (défaut), 'classic', 'oldstyle' and 'banking'
\moderncvcolor{blue}                               % autres couleurs : 'blue' (défaut), 'orange', 'green', 'red', 'purple', 'grey' and 'black'
%\nopagenumbers{}                                  % décommenter pour supprimer la numérotation automatique des pages pour les CVs de plus d'une page
\usepackage[utf8]{inputenc}                       % si vous n'utilisez pas xelatex ou lualatex, remplacer par le codage d'entrée que vous utilisez
\usepackage[scale=0.75,a4paper]{geometry}
\usepackage{babel}
%----------------------------------------------------------------------------------
%            informations personnelles
%----------------------------------------------------------------------------------
\firstname{Prénom}
\familyname{Nom}
\title{Titre du CV}                               % optionnel : supprimer ou commenter si non souhaité
\address{Numéro et rue}{Code postal et ville}{Pays} % optionnel : supprimer ou commenter si non souhaité; l'argument « Pays » peut être omis ou vide
\mobile{Numéro de portable}                          % optionnel : supprimer ou commenter si non souhaité
\phone{Numéro de téléphone}                           % optionnel : supprimer ou commenter si non souhaité
\fax{Numéro de fax}                             % optionnel : supprimer ou commenter si non souhaité
\email{Courriel}                               % optionnel : supprimer ou commenter si non souhaité
\homepage{Page Web personnelle}                         % optionnel : supprimer ou commenter si non souhaité
\extrainfo{Information supplémentaire}                 % optionnel : supprimer ou commenter si non souhaité
% \photo[64pt][0.4pt]{Image} % optionnel : décommenter si souhaité ; '64pt' est un exemple de hauteur que doit avoir la photo, 0.4pt est un exemple d'épaisseur que doit avoir le cadre qui l'entoure (à mettre à 0pt pour supprimer le cadre) et « Image » est le nom du fichier de la photo
\quote{Citation}                                 % optionnel : supprimer ou commenter si non souhaité
%
\begin{document}
\makecvtitle
\section{Formation}
\cventry{Année--Année}{Diplôme}{École}{Ville}{Mention}{Description}                      % les arguments 3 à 6 peuvent être laissés vides
\cventry{Année--Année}{Diplôme}{École}{Ville}{Mention}{Description}
\section{Expérience}
\subsection{Principale}
\cventry{Année--Année}{Emploi}{Employeur}{Ville}{}{Description générale d'au plus 1 ou 2 lignes}
\cventry{Année--Année}{Emploi}{Employeur}{Ville}{}{Description générale d'au plus 1 ou 2 lignes}
\subsection{Divers}
\cventry{Année--Année}{Emploi}{Employeur}{Ville}{}{Description générale d'au plus 1 ou 2 lignes}
\section{Langues}
\cvitemwithcomment{Langue 1}{Niveau}{Commentaire}
\cvitemwithcomment{Langue 2}{Niveau}{Commentaire}
\cvitemwithcomment{Langue 3}{Niveau}{Commentaire}
\section{Compétences informatiques}
\cvdoubleitem{Catégorie 1}{Commentaire}{Catégorie 4}{Commentaire}
\cvdoubleitem{Catégorie 2}{Commentaire}{Catégorie 5}{Commentaire}
\cvdoubleitem{Catégorie 3}{Commentaire}{Catégorie 6}{Commentaire}
\section{Centres d'intérêt}
\cvitem{Loisir 1}{Description}
\cvitem{Loisir 2}{Description}
\cvitem{Loisir 3}{Description}
\section{Extra 1}
\cvlistitem{Item 1}
\cvlistitem{Item 2}
\cvlistitem{Item 3}
\section{Extra 2}
\cvlistdoubleitem{Item 1}{Item 4}
\cvlistdoubleitem{Item 2}{Item 5}
\cvlistdoubleitem{Item 3}{Item 6}
\section{References}
\begin{cvcolumns}
  \cvcolumn{Catégorie 1}{Commentaire}
  \cvcolumn{Catégorie 2}{Commentaire}
  \cvcolumn{Catégorie 3}{Commentaire}
\end{cvcolumns}
\clearpage
%-----       letter       ---------------------------------------------------------
% recipient data
\recipient{DRH de l'entreprise}{Nom de l'entreprise\\Numéro et rue\\Code postal et ville}
\date{Date}
\opening{Chère madame, cher monsieur,}
\closing{Veuillez agréer,}
\enclosure{Pièces jointes}          %  utiliser l'argument optionnel pour spécifier un autre mot que "Enclosure", ou redéfinir \enclname
\makelettertitle
\makeletterclosing
\end{document}
